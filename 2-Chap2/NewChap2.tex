\chapter{Application context}\label{chp:2}

\minitoc

\clearpage

The most recent political answer brought to tackle environmental issues in Europe took form under the European Partnership for the Assesment of Risks from Chemicals (PARC). This partnership involves 28 different countries. This new project received a favourable assessment by the European Commission in January 2022 and has started on the 1st of May 2022. The main objectives of PARC are to promote European cooperation, to advance research, to increase knowledge about the risk assessment of chemicals and to train the corresponding methodological skills. Close cooperation between authorities and research will facilitate the translation of research results into regulatory practice. The French Agency for Food, Environmental and Occupational Health and Safety (ANSES) is not only the main french actor in this partnership but also the coordinator of the whole partnership. \\
This work was financed by the ANSES and aims at supporting the agency in its mission on the french territory. This chapter describes all the context and discussions that led to target specific goals. The chapter is organized as follows: we present the ANSES in \ref{chp:2:1}, the pesticides monitoring mission is detailed in \ref{chp:2:2}, we focus on the pesticides measures characteristics in \ref{chp:2:3}, we describe additionnal interesting sources on information on \ref{chp:2:4} and we fix the goals we aim for in \ref{chp:2:5}     

\section{ANSES presentation}\label{chp:2:1}

The ANSES was created in 2010 from the fusion of the French Food Safety Agency (AFSSA) and the French Agency for Environmental and Occupational Health Safety (AFSSET). It is a public administrative body reporting to the Ministries of Health, the Environment, Agriculture, Labour and Consumer Affairs.  

\subsection{Missions} 

\begin{itemize}
\item \textbf{Research activities:} the agency contributes to the progress of new scientific knowledge on the exposure of humans, animals, plants, and the environment to various hazards and risks and is tasked with improving their surveillance. Research topics focus on three areas: animal health and welfare, plant health, and food safety. ANSES is also involved in the development of new analytical methods and detection techniques to identify pathogens and contaminants, whether in the natural environment or in the production chain. This mission also regroups the activities health monitoring and alert. The agency takes part in epidemiological surveillance platforms on animal health, plants health and food chain safety. This mission also tasks ANSES to coordinate five different monitoring systems that covers the following areas: toxicovigilance, surveillance on food supplements, phytopharmacovigilance which will be further investigated in \ref{chp:2:2}, vetenary pharmacovigilance and surveillance and prevention of professionnal pathologies.      
\item \textbf{Risk assessment:} ANSES responds to society's questions about potential risks arising from the consumption of food, the use of certain products or technologies, professionnal activities, or pollution of various environmental compartments (e.g., air, water, or soil). Given the complex risks, the agency has developed a working methodology that brings together many disciplines to provide the most complete response possible. The risk and efficacy assessment of veterinary medicines and phytopharmaceutical products for human and animal health and the environment also falls within the agency's remit. The goal is to define management measures to handle such risks. In particular, ANSES is responsible for granting marketing authorizations for such products and thus also has the authority to withdraw products at the national level. 
\item \textbf{Public and environmental protection:} ANSES makes recommendations to support public debates and decisions. Its activities contribute to the implementation of effective preventive and protective measures on various societal issues such as health, biodiversity, and ethics. It also provides public access to reliable, independent, and multidisciplinary scientific information. The agency's role is to respond flexibly to already known or emerging, short- or long-term sanitary risks. In other words, the goal is to identify and make recommendations on all emerging signals as quickly as possible, even in times of crisis with scientific uncertainty. The task is then to reduce the degree of uncertainty when possible. Recommendations are based on all available knowledge, whether it was generated by the agency itself or by its partners. 
\end{itemize}



\subsection{Means of action}

\begin{itemize}
\item Les labos à elle, elle fait de la recherche
\item Le réseau de partenaires elle récupère et centralise plein de données
\end{itemize}

\subsection{Difference with other countries}



\section{Pesticids monitoring mission}\label{chp:2:2}

Phyto-pharmacovigilance is part of this type of study. It can be formally defined as a vigilance system that collects and analyzes monitoring data on phytopharmaceuticals (pesticides). The aim is to detect adverse effects associated with the use of these products as quickly as possible in order to protect the health of living organisms and ecosystems. Phytopharmacovigilance is an essential complement to the other tasks of the Agency, as one of its main tasks is to regulate by granting or denying authorizations for pesticide products\footnote{All their decision statements are available online \url{https://www.anses.fr/fr/decisions}.}.

\subsection{Emission sources}

\subsection{Diffusion in an environmental compartment}

\subsection{Potential contamination}

 

\section{Pesticids measurments specifities}\label{chp:2:3}

\subsection{Direct measurments information}

\subsubsection{Chemical precision limits in measurments}

\subsubsection{Irregular sampling}

\subsubsection{Spatio-temporal heterogeneity}

\subsection{Indirect measurments information}

\subsubsection{Surveys on farming pratices}

\subsubsection{Substances sales databank}



\section{Informations on environmental compartments}\label{chp:2:4}



\section{Objectives targeted}\label{chp:2:5}