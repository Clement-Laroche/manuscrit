

\chapter{Conclusion}

\section{Summary}


\section{Openings}

In this work we have seen that it is possible to extract information from data whose properties make modelling difficult. Further work is conceivable in the future. First, we have never addressed the issue of simultaneous monitoring of multiple substances. This is possible with the multivariate change point detection methods presented in Chapter 2, but these methods will always depend on heterogeneous spatiotemporal sampling. It can also be added that if different segmentations are obtained for different substances, there are ways to compare the positioning of these breaks. This topic is addressed, for example, in \cite{Cleynen}. 

Another starting point for future work is to try to optimise the placement of the stations and their sampling frequencies. It is obvious that a synchronous sampling rate of all stations would allow to observe the dynamics of the dispersion of the substance in space and time (in areas where there is only one emission source for the substance). Optimising the placement and sampling frequency of the stations is a similar issue to optimal design.