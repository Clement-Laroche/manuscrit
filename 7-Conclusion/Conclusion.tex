

\chapter{Conclusion and perspectives}

This thesis proposes an original method to extract spatio-temporal information from environmental data. We saw in Chapter \ref{chp:2} all the challenges that the ANSES is faced with. Time sampling heterogeneity and spatio-temporal heterogeneity together with specific censoring issues for concentration data demand the construction of sophisticated modelling and estimation methods and the design of relevant indicators to help experts in their monitoring tasks. A user-friendly implementation of these outcomes is also needed.

Chapters \ref{chp:3}, \ref{chp:4} and \ref{chp:5} cover the full description of the new method and data exploration process developed to address the above-mentioned issues.  Our approach is first driven by temporal detection. We aim at identifying time periods where the observed environmental concentrations are homogeneously distributed. Chapter \ref{chp:3} reviewed a selection of of change-point detection methods that seemed well adapted in our context, 

In Chapter \ref{chp:4}, we focused both on the effects of censoring on change-point detection methods, for models where some parameters are specific to each segment while others are shared by all segments. We derived an adapted optimisation method for our modelling framework. Chapter 5 provided the principles underlying our spatio-temporal detection method through a practical implementation on a real-life dataset, namely the concentration of Prosulfocarb in surface waters in the Centre-Val de Loire region between January 9, 2007 and April 8, 2022.   

We aggregated the temporal information at the regional scale to form series of daily maximum concentrations. The change point detection method proposed in Chapter \ref{chp:4} was then used on this series to reveal temporal regimes that proved to be good indicators of changes in farming activity.  In parallel, the stations monitoring surface water quality were modelled according to a graph. In this graph, the edges between the stations and their associated weights were computed from the spatial information on the river system in the region. The graph was then clustered to derive groups of stations, the resulting clusters constitute the aggregated spatial information. Finally, for each time regime, we used the available temporal and spatial information to propose a comparison of these clusters based on multi-criteria optimisation and identify the most anomalous ones. We hope that this detection constitutes a first useful indicator for experts in ANSES. 

Chapter \ref{chp:6} is the practical implementation of this procedure under the form of an \texttt{R} script for a preprocessing and precomputation step feeding an interactive Rshiny application. The design of this application results from discussions with the experts of the ANSES.

This thesis demonstrates that it is possible to extract information from spatio-temporal environmental data whose characteristics make modelling difficult. Some natural future developments emerge from this work. 

A first axis of investigation concerns developments aiming to improve the modelling and analysis of concentration data. We could further advance the analysis of pesticide concentrations by introducing a multivariate modelling framework in order to take simultaneous substances into account. Such methods are introduced in \cite{pickering2016changepoint}. This would broaden the scope of environmental monitoring to substance associations monitoring. The comparison of temporal change-points positions in different substances concentrations is possible with the works of \cite{Cleynen2014}. Observing similar change-points positions in different substances would imply a strong association in their use.  The evolution in time of the spatial distribution of anomalous clusters can also be another crucial point. We showed in Chapter \ref{chp:5} that the spatial distribution of Pareto levels did not seem to be uniform. Analysing the time series of the cluster Pareto levels could uncover additional information.

Irregular sampling, both in time and space, is another major issue in environmental data, therefore another axis could consist in improving the sampling procedure. There is currently a high spatio-temporal heterogeneity in the collected concentration data, which is damped through time and space aggregation in the analyses. This irregular sampling prevents from observing the dispersion dynamics of a substance in space and time. Providing a strategy for improving station locations and sampling schedules can be seen as an optimal design problem \citep{Mueller2011,Marsh2012}. This could prove to be a harsh task because optimal design problems would be highly constrained by the spatial structure underlying the station locations. For instance, the constraints on the location of stations monitoring air or surface waters quality are drastically different. The first case allows for almost any position in a given area, the latter is more constrained: stations have to be located on riverside (or at least near a surface water body).

Finally, from the implementation point of view, more work would be needed to go from the current prototype to a software allowing to handle larger datasets at the French national scale, and ultimately be able to compute and display indicators for several substances and any subregions. Integrating additional information such as the sales of the substance loaded in the application and the spatial distribution of the crops targeted by the substance would help in the monitoring mission.   



%This thesis provide a procedure to extract spatio temporal informations from environmental data. 
%We saw in Chapter \ref{chp:2} all the challenges that the ANSES is faced with. The mix of sampling heterogeneity and spatio-temporal heterogeneity combined with specific characteristic of censorship for concentration data demands the construction of sophisticated modeling and organized implementation. Chapters \ref{chp:3}, \ref{chp:4} and \ref{chp:5} cover full description of our exploratory process.     
%Our procedure is driven by temporal detection. We aim at identifying homogenous periods of time in the observed concentrations found in the environment. Chapter \ref{chp:3} reviewed the state of art of change-point detection methods that are used to find the homogeneous segments. In Chapter \ref{chp:4}, we focused on studying the effects of censorship on change-point detection methods and we derived a specific optimisation method that seemed suited for our modeling. 
%Chapter \ref{chp:5} provide a practical implementation of our spatio-temporal detection method. From the change-point detection constructed in Chapter \ref{chp:4}, we can deduce the farming activities from the resulting temporal regimes using aggregated information under the form of daily maximum concentration. Once we select a specific time regime, we use the environment structure to derive spatial information. The stations monitoring surface water quality were modeled according to a graph. The links of this graph are determined by the spatial information of the river system of the geographical area of study. Stations were clustered according to the graph structure. The resulting clusters constitute the aggregated spatial information. A comparison of these clusters is made to identify the most anomalous ones. 
%Chapter \ref{chp:6} is the practical implementation of this procedure into a \texttt{Rshiny} application. The design of this application is the results of discussion with the experts of the ANSES.  

%We saw that it is possible to extract information from data whose properties make modelling difficult. It seems that some natural future developments emerge from this work. 
%The first axis of investigation regroups all works that aim to improve the modeliing of concentration data. We can push pesticids analysis forward by introducing a multivariate modeling taking simultaneous substances into account. Such methods are introduced in \cite{pickering2016changepoint}. This broadens the scope of monitoring to substances associations. The comparison of temporal change-points positions in different substances concentrations is possible with the works of \cite{Cleynen2014}. Observing similar change-points positions in different substances would imply a strong association in use. The evolution of the spatial distribution of anomalous clusters in time can also be another crucial point. We showed in Chapter \ref{chp:5} that the spatial distribution of Pareto levels did not seem to be uniform. Analysing the time series of the clusters' Pareto levels could uncover additional informations.  
 
%Although many models and methods improvments can be done, it doesn't tackle the issue of sampling. Another axis of work would consist in building another sampling procedure. This procedure has to dampen the spatio-temporal heterogeneity in the collect of concentration data. This irregular sampling prevents from observing the dynamics of dispersion of a substance in space and time. The question of stations locations and sampling rythms can be seen as an optimal design problem \cite{Mueller2011,Marsh2012}. It can be noted that this could prove to be a harsh task because those methods are highly dependent on the spatial structure underlying the positions of the stations. For instance, the spatial structures underlying stations monitoring air or surface waters quality are drastically different. The first case allows for almost any position in a given area, the latter is more constrained: stations has to be on riverside (at least near a surface water body).  
