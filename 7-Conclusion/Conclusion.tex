

\chapter{Conclusion}

This thesis provide a procedure to extract spatio temporal informations from environmental data. 
We saw in Chapter \ref{chp:2} all the challenges that the ANSES is faced with. The mix of sampling heterogeneity and spatio-temporal heterogeneity combined with specific characteristic of censorship for concentration data demands the construction of sophisticated modeling and organized implementation. Chapters \ref{chp:3}, \ref{chp:4} and \ref{chp:5} cover full description of our exploratory process.     
Our procedure is driven by temporal detection. We aim at identifying homogenous periods of time in the observed concentrations found in the environment. Chapter \ref{chp:3} reviewed the state of art of change-point detection methods that are used to find the homogeneous segments. In Chapter \ref{chp:4}, we focused on studying the effects of censorship on change-point detection methods and we derived a specific optimiation method that seemed suited for our modeling. 
Chapter \ref{chp:5} provide a practical implementation of our spatio-temporal detection method. From the change-point detection constructed in Chapter \ref{chp:4}, we can deduce the farming activities from the resulting temporal regimes using aggregated information under the form of daily maximum concentration. Once we select a specific time regime, we use the environment structure to derive spatial information. The stations monitoring surface water quality were modeled according to a graph. The links of this graph are determined by the spatial information of the riveer system of the geographical area of study. Stations were clustered according to the graph structure. The resulting clusters constitute the aggregated spatial information. A comparison of these clusters is made to identify the most anomalous ones. 
Chapter \ref{chp:6} is the practical implementation of this procedure into a \texttt{Rshiny} application. The design of this application is the results of discussion with the experts of the ANSES.  

We saw that it is possible to extract information from data whose properties make modelling difficult. It seems that some natural future developments emerge from this work. 
The first axis of investigation regroups all works that aim to improve the modeliing of concentration data. We can push pesticids analysis forward by introducing a multivariate modeling taking simultaneous substances into account. Such methods are presented in Chapter \ref{chp:2} and introduced in \cite{pickering2016changepoint}. This broadens the scope of monitoring to substances associations. The comparison of change-points positions in different substances concentrations is possible with the works of \cite{Cleynen2014}. Observing similar change-points positions in different substances would imply a strong association in use. The evolution of the spatial distribution of anomalous clusters in time can also be another crucial point. We showed in Chapter \ref{chp:5} that the spatial distribution of Pareto levels didn't seem to be uniform. Analysing the time series of the clusters' Pareto levels could uncover additional informations.  
 
Although many models and methods improvments can be done, it doesn't tackle the issue of sampling. The second axis of work consists in building another sampling procedure. This procedure has to dampen the spatio-temporal heterogeneity in the collect of concentration data. This irregular sampling prevents from observing the dynamics of dispersion of a substance in space and time. The question of stations lcoations and sampling rythms can be seen as an optimal design problem \cite{Mueller2011,Marsh2012}. It can be noted that this could prove to be a harsh task because those methods are highly dependent on the spatial structure underlying the positions of the stations. For instance, the spatial structures uncerlying stations monitoring air or surface waters quality is drastically different. The first case allows for almost any position in a given area, the latter is more constrained: stations has to be on riverside (at least near a surface water body).  
