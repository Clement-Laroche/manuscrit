\chapter{Introduction}\label{chp:1}



Studies on environmental data befall the multidisciplinary domain of environmental science that regroups various fields such as: physics, biology, chemistry, geography, ecology. That important gathering of subjects induced a large collect of data coming from different sources of information. As stated in \cite{Manly2008}, the emergence of environmental statistics comes from the obvious fact that much of what is learned on the environment is based on numerical data. There are three broad types of areas of studies that we believe it is important to state: 
\begin{itemize}
    \item \textbf{Baseline studies} aim at documenting the present knowledge and how environmental processes operate. Future changes will be defined as any deviation from the standards identified by those studies.
    \item \textbf{Targeted studies} intend to characterize and assess the impact of planned or known changes (accidents, human activities). 
    \item \textbf{Regular monitoring} is designed to detect patterns such as variations, trends or changes in important parameters.  
\end{itemize}
In recent years, we can cite numerous applications that can be designated as environmental statistics and they span on a very large scope ranging from ice front change monitoring \cite{BUNCE_2018}, emerging marine diseases \cite{Harvell1999}, depletion of fossil fuels \citep{Hoeoek2013}, vegetation changes \citep{Zheng2021}, temperature evolution \citep{Shi2022}, extreme events occurence monitoring \citep{Zhao2010} and many more areas \citep{Ozgul2010,Mori2012}. 

This thesis proposes a statistical approach that aims at supporting the monitoring activity of phyto-pharmaceutical products. Monitoring the environmental pollution is of great interest for public authorities, important adverse health-effects being well documented nowadays \citep{khopkar2007,Marchant2018,NOUGADERE201432}. National health agencies are thus much concerned with monitoring ambient levels and quantifying the concentration of various pollutants in given environmental areas. 

Monitoring pollutant in the environment implies to use sensors at different locations and that perform samples in different moments in time. Hence, the collected data is spatio-temporal information. Modelling such data is a complex issue, due to several reasons, some intrinsic to the types of data under study, some specific to the data collection process implemented in different countries. Pollutant concentration levels are measured by sensors which have generally detection and quantification limits: the corresponding data are then left-censored. Secondly, the data is usually skewed to the right, with long tails hinting high concentrations. Thirdly, in numerous situations the data is irregularly sampled because of measurement practices, and is often multivariate, since various pollutant levels are monitored. Fourthly, pollution is monitored in various locations, each location possibly using different sensors, yielding a significant spatial heterogeneity. 

This thesis focuses on dealing with censored values and spatio-temporal heterogeneity. This demands a procedure that articulates different models and methods. The general principle is to find time periods and spatial areas where the informations are the most homogeneous possible using the most coherent datasets possible. Once these moments and zones are identified, we aim at detecting the most anomalous zones in these time periods. The manuscrit is organized as follows: 
\begin{itemize}
\item{\textbf{Chapter \ref{chp:2}}} is an introduction to the French national agency in charge of monitoring pollution data. A short description of the agency and its missions are given with an extensive description of the data available. Several datasets from different sources of information are useful to the analysis of environmental pollution. The last section presents a first example based on vizualisation techniques that allows to extract informations from these data sets.        
\item{\textbf{Chapter \ref{chp:3}}} makes an inventory of methods that are useful in the analysis of environmental pollution. The analysis of such data is difficult in presence of high spatio-temporal heterogeneity. We are looking to cut the time series into more homogeneous subdivisions. This objective befalls the change-point detection field in the litterature. We provide a review of such methods in this Chapter.    
\item{\textbf{Chapter \ref{chp:4}}} builds a specific parametric change-point detection method. We are looking for an adapted method for the problems we are facing. We study the effect of censorship to show that it does not prevent to find change-points in a signal. We also derive an optimization procedure that is suited to some modeling configuration. Simulation experiments are led to compare it with a state of the art non parametric method that is also adapted to censored data.        
\item{\textbf{Chapter \ref{chp:5}}} provide the spatial analysis of concentration data in the environment. It uses the results of the change-point detection method developped in Chapter \ref{chp:4} on the temporal dimension. Using spatial clustering and anomaly detection methods, we manage to extract some informations useful to assist experts in the environmental pollution monitoring mission.     
\item{\textbf{Chapter \ref{chp:6}}} describes the elaboration of an interactive application that displays the results of our procedure. This application serves an operationnal purpose. It is specifically designed for the experts working in that area of expertise  
\end{itemize}
