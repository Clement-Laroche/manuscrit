\section*{Résumé}

La mission de phyto-pharmacovigilance consiste à établir la surveillance de concentrations des produits phyto-pharmaceutiques dans des milieux environnementaux d'intérêt. Établir une telle surveillance n'est pas chose facile étant donné l'importante accumulation de données de natures différentes qui pourraient aider à une meilleure compréhension de la diffusion des substances surveillées. Les données de concentrations sont relevées par des stations de mesure réparties sur l'ensemble du territoire. Ce sont donc des données spatio-temporelles. De plus, les données relevées présentent plusieurs caractéristiques qui compliquent leur analyse comme de la censure, de l'hétérogénéité spatio-temporelle ou encore une forme particulière dans leur distribution empirique. On cherche donc à développer des méthodes adaptées à ces caractéristiques pouvant extraire des informations spatiales et temporelles anormales à fournir à une équipe d'expert de la phyto-pharmacovigilance.       

Deux contributions originales sont développées dans ce manuscrit de thèse. La première est une méthodologie en trois étapes dont la finalité est de détecter des clusters anormaux lors de périodes temporelles précises. Les périodes temporelles en question sont obtenues par détection de ruptures sur la série des concentrations agrégée à une certaine échelle temporelle. Une fois que l'on se place dans un segment temporel découlant de cette détection, on peut effectuer une comparaison de clusters spatiaux obtenus par clusterting hiérarchique. Cette comparaison peut être implémentée par optimisation multi-critère. On peut de cette manière détecter des clusters anormaux de manière contextuelle au segment temporel sélectionné. La méthode de détection de ruptures est spécialement adaptée aux caractéristiques des données de concentrations. Les performances de la détection de ruptures sont testées sur des données simulées. Elle est également comparée à une méthode de la littérature adaptée aux données censurées. La deuxième contribution de ce travail est une présentation interactive de l'ensemble des résultats obtenus par cette méthode sous la forme d'une application interactive \texttt{Rshiny}.  \\

\textbf{Mots-Clés :} phyto-pharmacovigilance, détection de ruptures, clustering, optimisation multi-critère, données spatio-temporelles, données censurées à gauche, hétérogénéité spatio-temporelle, application \texttt{Rshiny}.