\chapter{Introduction}\label{chp:1}

%This thesis is part of a series of works that investigate the pollution levels in different environmental fields. The aim is to help experts to summarise information from field measurements, to analyse these measurements and to warn of potential hazards. Monitoring the environmental pollution is of great interest for public authorities, important adverse health-effects being well documented nowadays \citep{khopkar2007,Marchant2018,NOUGADERE201432}. National health agencies are thus very concerned about monitoring ambient levels and quantifying the concentration of various pollutants in given environmental areas. 

%This thesis is the main result of a research and development agreement financed by the French Agency for Food, Environmental and Occupational Health and Safety. One of the agency's mission is to observe environmental data and provide tools to ensure public and environmental protection. An example of such tool would be a monitoring system that is designed to detect patterns such as variations, trends or changes in important parameters \citep{Manly2008}. More precisely, the parameters of interest in this thesis are concentration levels of pesticides.

%Hence, all discussions and works were conducted in close collaboration with the experts of the Phytopharmacovigilance Unit (UPPV). This unit collects and studies data of phytopharmaceutical products concentrations. Such data is collected by measuring stations that are positionned all over the territory. More precisely, this thesis focuses on the concentrations measures of a commonly used substance in France: the prosulfocarb. It is a herbicide used to protect wheat and barley from weeds. However, its presence in waters can be toxic for the aquatic fauna.       

%Monitoring the levels of such substance is very complex due to the way its concentrations are measured. Indeed, monitoring pollutant in the environment implies to use sensors at different locations that collect samples in different moments in time. Hence, the collected data is spatio-temporal information. Modelling such data is a complex issue, due to several reasons, some intrinsic to the characteristics of the collected data, some specific to the data collection process implemented in different countries. 

%Let us see some of these characteristics:
%\begin{itemize}
%\item Pollutant concentration levels are measured by sensors which have generally detection and quantification limits: the corresponding data are then left-censored.
%\item The data is usually skewed to the right, with long tails hinting high concentrations.
%\item In numerous situations the data is irregularly sampled because of measurement practices.
%\item The data is often multivariate, since various pollutant levels are monitored. 
%\item Pollution is monitored in various locations, each location possibly using different sensors, yielding a significant spatial heterogeneity.
%\end{itemize} 

%This thesis proposes a new approach to monitore concentrations of substances that present these characteristics. This demands a procedure that articulates different models and methods. The general principle is to find time periods and spatial areas where the information are the most homogeneous possible using the most coherent datasets possible. Once these moments and zones are identified, we aim at detecting the most anomalous zones in these time periods.  

%More precisely, we devise a change-point detection method that we use for modelling temporal heterogeneity, by assuming a piece-wise stationary distribution on series of concentration values, for a given time resolution. Clustering is then used for modelling the expected spatial homogeneity while integrating geographical constraints such as river networks, wind directions, etc. Conditionally to the temporal segment detected by the change-point procedure, and to the spatial cluster detected by the clustering procedure, one may analyse the data and identify contextual anomalies. The identification of anomalies is made with multiple-criteria decision analysis. 

%The work developped in this thesis led to several contributions such as two presentations in scientific conferences \footnote{Journées de Statistiques de la Société Française de Statistiques: \url{https://jds2021.sciencesconf.org/}}\footnote{International Conference on Operation Research: \url{https://rev-inv-ope.pantheonsorbonne.fr/2022-icor-conference}}, three presentations at the ANSES experts work groups and the one publication \citep{Laroche2022}. Furthermore, all the methods presented here were coded and stored in \texttt{Rcpp} scripts, the whole process to obtain monitoring results was centralized in a \texttt{R} script and a monitoring tool based on interactive programmation was implemented with a  \texttt{Rshiny} application.    

%The manuscrit is organized as follows: 
%\begin{itemize}
%\item{\textbf{Chapter \ref{chp:2}}} is an introduction to the French national agency in charge of monitoring pollution data. A short description of the agency and its missions are given with an extensive description of the data available. Several datasets from different sources of information are useful to the analysis of environmental pollution. The last section presents a first example based on vizualisation techniques that allows to extract information from these data sets.        
%\item{\textbf{Chapter \ref{chp:3}}} makes an inventory of change-point detection methods. Since extensive reviews on change-point detection methods already exist, we only focus on some specific elements that could be useful when applying such methods on environmental data. We cover the case where the number of changes in a signal is unknown. We provide search methods (or heuristics) that give an optimal solution to that problem. The last section discusses applications of change-point methods in environmental data that were previously published in the litterature. 
%\item{\textbf{Chapter \ref{chp:4}}} builds a specific parametric change-point detection method. We are looking for an adapted method for the problems we are facing. More precisely, we study the effect of censorship on change-point detection in a signal. Then, we discuss different optimization strategies when estimating the parameters of our model. We derive an iterative procedure to estimate both piece-wise constant parameters and stationnary parameters over time. Simulation experiments are led to compare it with a non parametric method that is also adapted to censored data.        
%\item{\textbf{Chapter \ref{chp:5}}} provides the case study of prosulfocarb concentrations in the Centre-Val de Loire french region. Both temporal and spatial analysis are made on this dataset. It uses the results of the change-point detection method developped in Chapter \ref{chp:4} on the temporal dimension. The spatial heterogeneity is handled using classical clustering methods. we manage to extract some useful information on the resulting geographical areas and time periods  with the final anomaly detection step. 
%\item{\textbf{Chapter \ref{chp:6}}} describes the elaboration of an interactive application implemented in \texttt{Rshiny} that displays the results of the procedure of Chapter \ref{chp:5}. This application is illustrated with the prosulfocarb dataset. Since the results of Chapter \ref{chp:5} are presented in a static way (e.g. using Figures), we can only present the results for a single detected time segment. The application allows to store the results for all time segments detected and to explore them in an interactive manner. This application serves an operationnal purpose. It is specifically designed for the experts working in that area of expertise  
%\end{itemize}


This thesis is the main result of a research and development agreement financed by the French Agency for Food, Environmental and Occupational Health and Safety (ANSES\footnote{Agence nationale de sécurité sanitaire de l’alimentation, de l’environnement et du travail}). One of the main mission of  ANSES is to observe environmental data and to provide tools to ensure public and environmental protection.

In particular, the national health agencies are very concerned about monitoring and quantifying the concentration of various pollutants in given environmental areas, since important adverse health-effects are well documented nowadays (\cite{khopkar2007,Marchant2018,NOUGADERE201432} for example). Our work aims to define news algorithms and statistical methods to control the environmental pollution, which  is of great interest for public authorities. 

The goal is to help experts to summarise and analyse information from field measurements, to analyse these measurements, to detect potential anomalies and to issue   alerts  if  it  is  necessary.

An example of such a tool could be for example a monitoring system that is designed to detect patterns such as variations, trends or changes in relevant series of environmental measures over time and space \citep{Manly2008}.

More precisely, this thesis focuses on the concentrations measures of a commonly used substance in France: the prosulfocarb. It is a herbicide used to protect wheat and barley from weeds. However, its presence in waters can be toxic for the aquatic fauna.  

Hence, all discussions and works were conducted in close collaboration with the experts of the Phytopharmacovigilance Unit (UPPV). This unit gathers and studies data of phytopharmaceutical products concentrations. Such data is collected in measuring stations that are positioned all over the French territory. 

Indeed, monitoring pollutant in the environment involves using sensors at different locations that collect samples at different points in time. Hence, the collected data collected is spatio-temporal information. 

Modelling such data is a complex issue, due to several reasons, some intrinsic to the characteristics of the collected data, some specific to the data collection process implemented in the different countries.

Let us see some of these characteristics:
\begin{itemize}
\item Pollutant concentration levels are measured by sensors which have generally detection and quantification limits: the corresponding data are then left-censored.
\item The data is usually skewed to the right, with long tails hinting suggesting high concentrations.
\item In many situations data is irregularly sampled due to measurement practices (measurements require human interventions).
\item Pollution is monitored at various locations, with each location possibly using different sensors, resulting in significant spatial heterogeneity.
\end{itemize}

This thesis proposes a new approach to deal with time series that present these characteristics. It requires a procedure that combines different models and methods. The basic principle is to find time periods and spatial areas where the information are the most homogeneous possible, using the most coherent datasets possible. Once these moments and these zones have been identified, we seek to detect the most anomalous zones in these periods of time. 

More precisely, let us summarize the procedure we define, which is the main contribution of our work. 

First we define a change-point detection method to model the temporal heterogeneity, by assuming a piece-wise stationary distribution of the series of concentration values, for a given time resolution. 

Clustering is then used for modelling the expected spatial homogeneity while integrating geographical constraints such as river networks, wind directions, etc. 

Finally conditionally to the temporal segment detected by the change-point procedure, and to the spatial cluster built by the clustering procedure, we can analyse the data and identify contextual anomalies. Then the identification of anomalies is made with using multiple-criteria decision analysis.

The manuscrit is organized as follows:
\begin{itemize}
\item{\textbf{Chapter \ref{chp:2}}} is a presentation of the French national agency in charge of monitoring pollution data. A short description of the agency and of its missions are given with an extensive description of the available data derived from different sources of information that are useful for the analysis of environmental pollution. The last section presents a first example based on vizualisation techniques that allows to extract information from these data sets.
\item{\textbf{Chapter \ref{chp:3}}} makes an inventory of change-point detection methods. Since extensive reviews on change-point detection methods already exist, we only focus on some specific elements that could be useful when applying such methods to environmental data. We cover the case where the number of changes is unknown. We provide search methods (or heuristics) that give an optimal solution to that problem. The last section discusses applications of change-point methods in environmental data that were previ- ously published in the litterature.
\item{\textbf{In Chapter \ref{chp:4}}} we build a specific parametric change-point detection method. Our contribution consists in defining an adapted method for the problems we are facing. More precisely, we study the effect of censorship on the change-point detection. Then, we discuss different optimization strategies to estimate the parameters of the model. We define an iterative procedure to estimate at the same time piece-wise constant parameters and stationary parameters over time. Simulation experiments are led to compare the proposed method with a non parametric method that is also adapted to censored data.
\item{\textbf{Chapter \ref{chp:5}}} provides the case study of prosulfocarb concentrations in the Centre-Val de Loire French region. Both temporal and spatial analysis are made on this dataset. It uses the results of the change-point detection method developped in Chapter \ref{chp:4} on for the temporal dimension. 
The spatial heterogeneity is handled using classical clustering methods. We manage to extract some useful information from the resulting geographical areas and time periods by adding a final anomaly detection step. Our contribution results in the definition of a method which deals with rough data to finally provide alert about potential aniomalies. 
\item{\textbf{Chapter \ref{chp:6}}} describes the development of an interactive application implemented in Rshiny that displays the results of the procedure defined in Chapter \ref{chp:5}. This application is illustrated by the prosulfocarb dataset. Since the results of Chapter \ref{chp:5} are presented in a static way (e.g. using Figures), first of all we can only present the results for a single detected time segment. So the application allows to store the results for all the time segments detected and to explore them interactively. This application serves an operational purpose. It is specifically designed for the experts working in that area of expertise
\end{itemize}

{\Large\textbf{Contributions}}
\begin{itemize}
\item \textbf{Articles:} \\ 
\bibentry{Laroche2022} \\
\bibentry{JdS} \\
\bibentry{ICOR}
\item \textbf{Script:} \\
Monitoring procedure with prosulfocarb dataset: \url{https://github.com/Clement-Laroche/special-issue}
\item \texttt{Rshiny} \textbf{application:} \\
\url{https://github.com/Clement-Laroche/Application_ANSES}
\end{itemize}
