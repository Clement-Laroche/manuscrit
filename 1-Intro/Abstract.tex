\section*{Abstract}

The task of phytopharmacovigilance is to establish monitoring of concentrations of phytopharmaceuticals in relevant environmental media. Establishing such monitoring is not an easy task given the large accumulation of data of different types that could help to better understand the distribution of the monitored substances. Concentration data are collected from monitoring stations distributed throughout the area. It is therefore spatio-temporal data. Moreover, the collected data have several characteristics that complicate their analysis, such as censoring, spatial and temporal heterogeneity and a particular form of their empirical distribution. We are therefore trying to develop methods adapted to these characteristics that can extract anomalous spatial and temporal information to make it available to a team of phyto-pharmacovigilance experts.

Two original contributions are developed in this manuscript. The first is a three-step methodology aimed at detecting anomalous clusters in specific time periods. The time periods in question are identified by detecting change points in the aggregate concentration series on a given time scale. In a time segment resulting from this detection, a comparison of the spatial clusters obtained by hierarchical clustering can be performed. This comparison can be performed by multi-criteria optimisation. Clusters are highlighted as anomalous if they have a high Pareto front value. Anomaly detection is thus contextual to the selected temporal segment. The break detection method is specifically adapted to the characteristics of the concentration data. The performance of the break detection is tested on simulated data. It is also compared with a method from the literature adapted to censored data. The second contribution of this work is an interactive presentation of all results obtained with this method in the form of an interactive application \texttt{Rshiny}. \\

\textbf{Keywords :} phyto-pharmacovigilance, change point detection, clustering, multicriterion analysis, spatio-temporal data, left censored data, spatio-temporal heterogeneity, \texttt{Rshiny} application.